\documentclass[russian]{lecture-notes}

\usepackage[final]{graphicx} % возможность вставлять изображения в текст
\usepackage{subcaption} % пакет нужен, чтобы можно было ставить несколько рисунктов рядом     

\usepackage{amssymb} % для знака пустого множества
\usepackage{amsfonts} % для букв с двойными штрихами (знак натуральных чисел)

\usepackage{tikz}  % для рисования графов
\usetikzlibrary{graphs}

\usepackage{cancel} % для зачеркивания слова

\title{Математическая логика и теория алгоритмов}
\lecturer{Поcов Илья Александрович}
\notesauthor{Блюдин Андрей}

\begin{document}
	\maketitle

\begin{sloppypar}

\tableofcontents

\section{Математическая логика}

\subsection{Исчисление высказываний}

\subsubsection{Основные понятия}

\begin{definition} 
	Логическая функция ~--- это множество из 2 элементов. Также, логической функцией называют множество логических значений
		
	$B = \{0, 1 \}$, где 0 ~--- это ложь (false), а 1 ~--- это истина (true)
\end{definition}

\begin{definition} 
	Логическая функция от $n$ переменных
	$$f : B^n \rightarrow B$$ 
\end{definition}

\begin{remark}
    Часто логические функции вводят как перечисление возможных аргументов и значений функции при этих аргументах
\end{remark}

\begin{example}
	Введем функцию $f(x,y)$
	
	\begin{table}[h!]
		\centering
		\begin{tabular}{|c|c|c|cl}
			\cline{1-3}
				x & y & f(x,y) & &  \\ \cline{1-3}
				0 & 0 & 0      &  &  \\ \cline{1-3}
				0 & 1 & 1      &  &  \\ \cline{1-3}
				1 & 0 & 1      &  &  \\ \cline{1-3}
				1 & 1 & 1      &  &  \\ \cline{1-3}
		\end{tabular}
		\caption{Таблица истинности для $f(x,y)$}
	\end{table}
	
	Эту же функцию можно задать функцией $f(x,y) = max(x,y)$
\end{example}

\begin{proposition}
    Функция от $n$ переменных может быть $f(x_1, x_2, x_3, \dots, x_n)$
    
    \begin{table}[h!]
  	  	\centering
		\begin{tabular}{|c|c|c|c|l|}
			\hline
			$x_1$ & $x_2$ & ... & $x_n$                   & f($x_1, x_2, \dots, x_n$) \\ \hline
			0    & 0    & ... & 0                      & 0 или 1                  \\ \hline
			...  & ...  & ... & ...                    & 0 или 1                  \\ \hline
			1    & 1    & ... & \multicolumn{1}{l|}{1} & 0 или 1                  \\ \hline
		\end{tabular}
		\caption{Таблица истинности для $f(x_1,x_2, \dots, x_n)$}
	\end{table}
		
	При этом количество всех возможных наборов аргументов равняется $2^n$, а количество всех возможных функций при всех возможных наборах аргументов равняется $2^{2^n}$
\end{proposition}

\begin{corollary}
    Посчитаем количество таких функий для разных $n$
    
    $n = 1 \quad 2^2 = 4$ функций $f(x)$
    
	$n = 2 \quad 2^{2^2} = 16$ функций $f(x,y)$

	$n = 3 \quad 2^{2^3} = 2^8 = 256$ функций $f(x,y,z)$
\end{corollary}

\subsubsection{Функции от 1 переменной (их определения)}

\begin{example}
	Перечислим все возможные функции от 1 переменной
	
	\begin{table}[h!]
	\centering
		\begin{tabular}{|c|c|c|c|l|}
			\hline
			$x$ & $f_1(x)$ & $f_2(x)$ & $f_3(x)$ & $f_4(x)$ \\ \hline
			0 & 0       & 0       & 1       & 1       \\ \hline
			1 & 0       & 1       & 0       & 1       \\ \hline
		\end{tabular}
	\end{table}
	
	Данные функции имеют значение:
	
	$f_1(x) = 0$ ~--- функция 0
	
	$f_2(x) = x$ ~--- функция $x$
	
	$f_3(x) = !x, \bar x, \neg x$, not $x$ ~--- функция отрицания (не $x$)
	
	$f_4(x) = 1$ ~--- функция 1
	
\end{example}

\subsubsection{Функции от 2 переменных (их определения)}

\begin{example}
	Перечислим все возможные функции от 2 переменных
	
	\begin{table}[h!]
	\centering
		\begin{tabular}{|c|c|c|c|c|c|c|c|c|c|}
			\hline
			$x$ & $y$ & $f_1(x)$ & $f_2(x)$ & $f_3(x)$ & $f_4(x)$ & $f_5(x)$ & $f_6(x)$ & $f_7(x)$ & $f_8(x)$ \\ \hline
			0 & 0 & 0       & 0       & 0       & 0       & 0       & 0       & 0       & 0       \\ \hline
			0 & 1 & 0       & 0       & 0       & 0       & 1       & 1       & 1       & 1       \\ \hline
			1 & 0 & 0       & 0       & 1       & 1       & 0       & 0       & 1       & 1       \\ \hline
			1 & 1 & 0       & 1       & 0       & 1       & 0       & 1       & 0       & 1       \\ \hline
		\end{tabular}
		\caption{Таблица истинности для $f(x,y)$}
	\end{table}
	
	Продолжение:
	
	\begin{table}[h!]
	\centering
		\begin{tabular}{|c|c|c|c|c|c|c|c|c|c|}
			\hline
			$x$ & $y$ & $f_9(x)$ & $f_{10}(x)$ & $f_{11}(x)$ & $f_{12}(x)$ & $f_{13}(x)$ & $f_{14}(x)$ & $f_{15}(x)$ & $f_{16}(x)$ \\ \hline
			0 & 0 & 1       & 1       & 1       & 1       & 1       & 1       & 1       & 1       \\ \hline
			0 & 1 & 0       & 0       & 0       & 0       & 1       & 1       & 1       & 1       \\ \hline
			1 & 0 & 0       & 0       & 1       & 1       & 0       & 0       & 1       & 1       \\ \hline
			1 & 1 & 0       & 1       & 0       & 1       & 0       & 1       & 0       & 1       \\ \hline
		\end{tabular}
		\caption{Таблица истинности для $f(x,y)$}
	\end{table}
	
	\textbf{Перечислим основные значения функций:}
	
	$f_2(x,y)$ ~--- это конъюнкция или "лочическое и"  или логическое умножение ($xy, x\&y, x \wedge y$)
	
	$f_7(x,y)$ ~--- это исключающее или ($x+y, x XOR y, x \oplus y$), также данную функцию можно ассоциировать как $(x+y) mod 2$
	
	$f_8(x,y)$ ~--- это логическое или, но ее можно также записать как $max(x,y)$  ($x|y, x \lor y$)
	
	$f_{10}(x,y)$ ~--- это эквивалентность ($x \Leftrightarrow y, x \equiv y, x == y$)
	
	$f_{14}(x,y)$ ~--- это импликация ($x \Rightarrow y, x \rightarrow y$)
	
	Импликация работает так, что истина следует из чего угодно:
	
	лешия не существует $\Rightarrow$ русалок не существует = 1 (1 $\Rightarrow$ 1 = 1)
	
	допса скучная $\Rightarrow$ русалок не существует = 1 (0 $\Rightarrow$ 1 = 1)
	
	русалки существуют $\Rightarrow$ драконы существуют = 1 (0 $\Rightarrow$ 0 = 1)	

	 x $\Rightarrow$ y = 0 только если x = 1, а y = 0

	$f_{12}(x,y)$ ~--- это обратная импликация ($x \Leftarrow y = y \Rightarrow x$)
	
	$f_{9}(x,y)$ ~--- стрелка Пирса ($x \downarrow y = \overline{x \lor y}$)

	$f_{15}(x,y)$ ~--- штрих Шеффера ($x | y = \overline{xy}$)
	
	$f_{3}(x,y)$ ~--- запрет по y ($x > y = \overline{x \Rightarrow y}$)

	$f_{1}(x,y)$ ~--- 0

	$f_{4}(x,y)$ ~--- $x$
	
	$f_{5}(x,y)$ ~--- запрет по x ($x < y = \overline{x \Leftarrow y}$)

	$f_{6}(x,y)$ ~--- $y$
	
	$f_{11}(x,y)$ ~--- не y ($\neg y$)
	
	$f_{13}(x,y)$ ~--- не x ($\neg x$)
	
	$f_{16}(x,y)$ ~--- 1

\end{example}

\begin{definition} 
	Логические выражения ~--- способ задания логических функций с помощью переменных, цифр 0 или 1 и операций:
	
	$\cdot \quad \lor \quad \Rightarrow \quad  \Leftrightarrow \quad + \quad \equiv \quad | \quad \downarrow \quad < \quad >$ 
\end{definition}

\begin{example} 
	Примеры логических выражений:

	$(x \lor y) = $
	
	$(x \Rightarrow yz) \lor (y \equiv z)$
	
	$(0 \Rightarrow x) \lor (1 \Rightarrow y)$
\end{example}

\begin{definition} 
	Значения логического выражения можно записать \textbf{Таблицей истинности}
\end{definition}

\begin{example}
	$f(x, y, z) = (x \lor y)z$
	
	\begin{table}[h!]
		\centering
		\begin{tabular}{|c|c|c|c|}
			\hline
			x & y & z & f(x,y,z) \\ \hline
			0 & 0 & 0 & 0        \\ \hline
			0 & 0 & 1 & 0        \\ \hline
			0 & 1 & 0 & 0        \\ \hline
			0 & 1 & 1 & 1        \\ \hline
			1 & 0 & 0 & 0        \\ \hline
			1 & 0 & 1 & 1        \\ \hline
			1 & 1 & 0 & 0        \\ \hline
			1 & 1 & 1 & 1        \\ \hline
		\end{tabular}
	\end{table}
\end{example}

\begin{remark}
	Порядок строчек в таблеце истинности может быть любым, но лучше использовать как у двоичных чисел    
\end{remark}

\begin{proposition}
    Таблицы истинности часто считают постепенно
	
	\begin{table}[h!]
		\centering		
		\begin{tabular}{|c|c|c|c|l|}
			\hline
			x & y & z & $x \lor y$ & $(x \lor y)z$  \\ \hline
			$\dots$ & $\dots$ & $\dots$ & $\dots$ & $\dots$ \\ \hline
			$\dots$ & $\dots$ & $\dots$ & $\dots$ & $\dots$ \\ \hline
		\end{tabular}
	\end{table}
\end{proposition}

\subsubsection{Приоритеты операций}
	
	$$\neg$$
	
	$$\cdot$$
	
	$$\lor$$
	
	$$+ \quad \equiv$$
	
	$$\Rightarrow \quad \Leftarrow$$
	
	$$| \quad \downarrow \quad < \quad >$$
	
\begin{example}
	Примеры приоритетов операций:
	
	$\neg x \lor y = (\neg x) \lor y$
	
	$x \lor y z = x \lor (y z)$
	
	$x \Rightarrow y \lor z = x \Rightarrow (y \lor z)$
\end{example}	

\subsubsection{Алгебраические преобразования логических выражений}

\begin{definition} 
	Алгебраические преобразования логических выражений ~--- изменяем выражения по правилам, обычно в сторону упрощения
\end{definition}

\begin{example}
	$(0 \Rightarrow x) \lor (1 \Rightarrow y) = 1 \lor (1 \Rightarrow y) = 1$
\end{example}

\begin{proposition*}
    $$\overline{\overline{x}} = x$$
    
    \textbf{Доказательство:}
    
    \begin{table}[h!]
		\centering		
		\begin{tabular}{|c|c|c|}
			\hline
			$x$ & $\overline{x}$ & $\overline{\overline{x}}$ \\ \hline
			0 & 1 & 0 \\ \hline
			1 & 0 & 1 \\ \hline
		\end{tabular}
	\end{table}

\end{proposition*}

\begin{proposition*}
    При $\lor$:
    
    $$1 \lor x = 1$$
    
    $$0 \lor x = x$$
    
    $$x \lor y = y \lor x$$
\end{proposition*}

\subsubsection{Таблица эквивалентных логических выражений}

\begin{proposition}
    $x \lor y = y \lor x $ - симметричность
    

    $x \lor 0 = x$
    

    $x \lor 1 = 1$
    

    $x \lor x = x$
    

    $x \lor \overline{x} = 1 $
        
        
        $\qquad$ \textbf{Доказательство:}
        
        
    \begin{table}[h!]
		\centering		
		\begin{tabular}{|c|c|c|}
			\hline
			$x$ & $\overline{x}$ & $x \lor \overline{x}$ \\ \hline
			0 & 1 & $0 \lor 1 = 1$ \\ \hline
			1 & 0 & $1 \lor 0 = 1$ \\ \hline
		\end{tabular}
	\end{table}
        
    $xy = yx $

    $x*0 = 0$
    
    $x*1 = x$

    $x*x = x$

    $x* \overline{x} = 0$

    $x+y = y + x$
    
    $x + 0 = x$

    $x + 1 = \overline{x}$

    $x + x = 0$

    $x + \overline{x} = 1 $

\end{proposition}

\begin{proposition}
    
    $x \lor (y\lor z) = (x \lor y ) \lor z$ - ассоциативность
    
    Ассоциативность означает,что порядок скобок не важен 
    
    \begin{example} $x \Rightarrow y \neq y \Rightarrow x$ - не симметричная функция
    
    $$$$
    
        \textbf{Доказательство: }
        
    \begin{table}[h!]
        \centering	
        \begin{tabular}{|l|l|l|}
            \hline
            x y & $x \Rightarrow y$ & $y \Rightarrow x$ \\ \hline
            0 0 & 1 & 1 \\ \hline
            0 1 & 1 & 0 \\ \hline
            1 0 & 0 & 1 \\ \hline
            1 1 & 1 & 1 \\ \hline
        \end{tabular}
    \end{table}
    
    \end{example}
    
    \begin{remark}
        $x \Rightarrow y \neq y\Rightarrow x$
    \end{remark}
        
$x \Rightarrow 0 = \overline{x}$

$0 \Rightarrow x = 1$
 $$$$

\textbf{Доказательство: }

\begin{table}[h!]
\centering
\begin{tabular}{|l|l|}
\hline
x & $x \Rightarrow 0$ \\ \hline
0 & $0 \Rightarrow 0 = 1$ \\ \hline
1 & $1 \Rightarrow 0 = 0$ \\ \hline
\end{tabular}
\end{table}

$x\Rightarrow 1 = 1$

$1 \Rightarrow x = x$

$x \Rightarrow x = 1$

$x \Rightarrow \overline{x} = \overline{x}$

$\overline{x} \Rightarrow x = x$

$\overline{x} \Rightarrow y \Rightarrow z$ договоримся, что это $x \Rightarrow y (y\Rightarrow z) \neq (x \Rightarrow y) \Rightarrow z$

\begin{table}[h!]
\centering
\begin{tabular}{|l|l|l|l|l|}
\hline
x y z  & $x \Rightarrow y$ & $y \Rightarrow z$ & $x \Rightarrow (y \Rightarrow z)$ & $(x \Rightarrow y) \Rightarrow z$ \\ \hline
0 0  0 & 1               & 1                 & 1                                 & 0                                 \\ \hline
0 0 1  & 1               & 1                 & 1                                 & 1                                 \\ \hline
0 1 0  & 1               & 0                 & 1                                 & 0                                 \\ \hline
0 1 1  & 1               & 1                 & 1                                 & 1                                 \\ \hline
1 0 0  & 0               & 1                 & 1                                 & 1                                 \\ \hline
1 0 1  & 0               & 1                 & 1                                 & 1                                 \\ \hline
1 1 0  & 1               & 0                 & 0                                 & 0                                 \\ \hline
1 1 1  & 1               & 1                 & 1                                 & 1                                 \\ \hline
\end{tabular}
\end{table}

$x \Leftrightarrow y = y \Leftrightarrow x$

$x \Leftrightarrow 0 = \overline{x}$

$x \Leftrightarrow 1 = x$

$x \Leftrightarrow x =1$

$x \Leftrightarrow \overline{x} = 0$

$x \Leftrightarrow (y\Leftrightarrow z) = (x \Leftrightarrow y) \Leftrightarrow z$ - ассоциативно

\end{proposition}

\begin{proposition}

Дистрибутивность

$(x \lor y) z = xz \lor yz$

$(x+y)z = xz + yz$ по таблице истинности

$(x \& y ) \lor z$ ($xy \lor z = (x \lor z)(y \lor z)$

$(x \lor y) \& z = (x \& z) \lor (y \& z)$

$(x \& y) \lor z = (x \lor z) \& (y\lor z)$

\begin{remark}

$ $
    $(x_{1} \lor x_{2} \lor x_{3}) (y_{1} \lor y_{2}) = (x_{1} \lor x_{2} \lor x_{3})y_{1} \lor (x_{1} \lor x_{2} \lor x_{3})y_{2} = x_{1}y_{1} \lor x_{2}y_{1} \lor x_{3}y_{1} \lor x_{1}y_{2} \lor x_{2}y_{2} \lor x_{3}y_{2}$

    $xy \lor z = (x \lor z)(y\lor z) = xy \lor xz \lor zy \lor zz = xy \lor  xz \lor zy \lor z = xy \lor xz \lor zy \lor z*1 = xy \lor z(x\lor y \lor 1) = xy \lor z$ сошлось
    
\end{remark}

$x+y = \overline{x\Rightarrow}y$ - смотри Таблицу истинности

$(x\Rightarrow y)(y\Rightarrow x) = x \Rightarrow y$

\subsubsection{Многочлены Жегалкина}

    \begin{remark}
    
        Одну и ту же функцию можно записать по разному.
    
    \end{remark}
    
    В алгебре: $f(x) = 1 +x = x + 1 = x + 5 - 4 = sin(x-x) + x = ...$
    
    В логике: $f(x,y) = x \lor y = x \lor y \lor 0 = (x \lor y)(\overline{y} \lor y = x\overline{y} \lor y$ (= - дистрибутивность)
    
    \textbf{Многочлены Жегалкина для логической формулы}
    
    
    \begin{definition}
        $f(x_{1}......x_{n}$) - это многочлен с переменными xi, конспектами 0,1 и со степенями переменных $\leqslant$ 1. Это многочлены от xi $\mathbf{Z_{2}}$
    \end{definition}
    
    \begin{example}
    
        $f(x,y,z) = 1 + x + yz + xyz$
        
        $1 + x \qquad \qquad xy+xyz$
        
        $1 + xy$
    \end{example}
    
    Не многочлены
    
    $1 + x + (y \lor z)$
    
    $1 + x + z^{2}$ нельзя степень 2
    
    
    \begin{remark}
        В общем случае многочлен от 1 переменной ($а_{i} =$ 0 или 1)
        
        $a_{0} + a_{1}x$
        
        от 2ух: $a_{0} + a_{1}x+a{2}y+a{3}xy$
        
        от 3ех: $a_{0} + a_{1}x + a_{2}y + a_{3}z + a_{4}xy + a_{5}xz + a_{6}yz + a_{7}xyz$
        
        \end{remark}
        
        В общем случае $f(x1....x_{n}$) $a_{0} + a_{1}x_{1} + ... + a_{n}x_{n} + ax_{1}x_{2} + ax_{1}x_{3}$ + ... (все пары переменных) + $ ax_{1}x_{2}x_{3}+ax_{1}x_{3}x_{2} \leftarrow $все тройки перменных$ + ax_{1}x_{2}x_{3}...x_{n}$
        
        \begin{definition}
            $\forall f(x_{1}...x_{n})$ - логические функция $\exists!$ многочлен Жегалкина $g(x_{1}...x_{n}) : f = g$
        \end{definition}
        
        \begin{remark}
            Всего 4 функции от 1ой переменной
            
            $f(x) = 0 = \overline{x} = 0 + 0x$
            
            $f(x) = 1 = 1 = 1 + 0x$
            
            $f(x) = x = x = 0 + 1x$
            
           $f(x) = \overline{x} = 1 + x = 1 + 1x$
        \end{remark}
    
    \textbf{Докозательство:}
    
        \begin{definition}
            Разные многочлены - это разные логические функции т.е. $f(x_{1} ... x_{n} = a_{0} + ... + a_{1}x_{1} ... x_{n}$
            
            $g(x_{1}...x_{n}) = b_{0} + ... + bx_{1} ... x_{n}$
            
            $\exists!$ : $a_{i} \neq b_{i}$ различающийся
            
        \end{definition}
        
        \qquad \textbf{Доказательство:}
        
        Возьмем индекс с самым большик количеством переменных
        
        $f(x,y,z) = 1 + x + xy + xyz = ... + 1x + Dy + Dz + 1xy$
        
        $g(x,y,z) = 1 + y + z + xyz ... + Dx + 1y + 1z$
        
        для переменных этого слагаемого подставим 1 0xy 
        
        для остальных переменных : 0
        
        [ В примере $x=1,y=0,z=0 : f(1,0,0) и g(1,0,0)$ ]
        
        и в f и в g все другие слагаемые равны 0
        
            Теперь f(...) и g(...) 
            
                $f(...) = a_{i}x{1}x_{2}x_{3} \neq b_{i}x_{1}x_{2}x_{3}$ $\Rightarrow f(x_{1}...x_{n}$) $\neq y$ 
                
    \qquad \textbf{Доказательство:}
    
    Проверим, что многочленов Жегалкина столько, сколько функций:
    
        Посчитаем
        
        $a_{0} + a_{1}x_{1}+ ... + a_{1}x_{1}x_{2}...x_{n}$
        
        Сколько слагаемых:
        
        1) 1 слагаемых без переменных
        
        n слагаемых с переменной
        
        \qquad $a_{1}x_{1} + ... + a_{n}x_{n}$
        
        $C_{n}^{2}$ - слагаемых с 2 - мя переменными
        
        $С_{n}^{3}$ - слагаемых с 3 - мя переменными
        
        $C_{n}^{n}$ - слагаемых с n переменными
        
        Всего : $ C_{n}^{0} + C_{n}^{1} + C_{n}^{2} + ... C_{n}^{n} = 2^{n}((1+1)^{2}) $
        
    \begin{example}
            
    $a_{0} + a_{1}x$ - 2 слагаемых
            
    $a_{0} + a_{1}x + a_{2}y+a_{3}xy - 2^2 = 4$ слагаемых
    
    2) Все слагаемых имею вид: $x_{1},x_{2},x_{3}...x_{n}$ (0 или 1) - $2^n$ слагаемых
    
    Итого: многочлен Жегалкина от n переменных 
    \end{example}
    
    
    \begin{problem}
    Сколько разных многочленов?
    
    Это столько же, сколько логический функций
    
    Итог: 
    
    \textbf{Следствие:} Любая логическая функция может быть представлена в виде многочлена Жегалкина
    
    \end{problem}
    
    \begin{example}
    
        $f(x,y) = x \lor y$
        
        $f(x,y) = x*y$ - уже многочлен Жегалкина
        
        \textbf{Метод неопределенных коэффициентов:}
        
        
        Подберем $x \lor y = a_{0} + a_{1}x + a_{2}y + a_{3}xy$
        
        $f(0,0) = 0$
        
        $f(0,0) = a_{0} + a_{1}*0 + a_{2}*0 + a_{3}...$
        
        $f(1,0) = 1 \lor 0 = 1$
        
        $f(1,0) = a_{0} + a_{1} = a_{1}$ ($a_{0} = 0, \Rightarrow a_{1} = 1$
        
        $f(0,1) =$ аналогично $\Rightarrow a_{1} = 1$
        
        $f(x,y) = x+y + a_{3}xy$
        
        $f(1,1) = 1 \lor 1 = 1$
        
        $f(1,1) = 1 + 1 + a_{3} = 0 + a_{3} = a_{3}$, $a_{3}=1$
        
        Ответ: $x\lor y = x+y+xy$
        
        
    \end{example}

\end{proposition}

\end{sloppypar}

\end{document}